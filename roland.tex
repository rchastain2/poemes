\documentclass[a4paper,14pt,french]{extarticle}

\usepackage{fontspec}
\usepackage[french]{babel}
\usepackage{microtype}
\usepackage{verse}

\setmainfont{EBGaramond}
[
  Extension      = .otf ,
  UprightFont    = *-Regular,
  ItalicFont     = *-Italic,
  BoldFont       = *-Bold,
  BoldItalicFont = *-BoldItalic,
  Numbers        = Lowercase,
  Ligatures      = Discretionary,
  Style          = Swash
]
% https://texnique.fr/osqa/questions/8423/lualatex-et-ligatures

\settowidth{\versewidth}{Félons païens furent mal inspirés de venir aux défilés :}

\usepackage[margin=16mm,bmargin=24mm]{geometry}

%\pagestyle{empty}

\begin{document}
%\thispagestyle{empty}

\poemtitle{L'entêtement de Roland.}

\small
\textit{Alors que l'armée de Charlemagne, revenant d'Espagne, traverse les Pyrénées, son arrière-garde est attaquée par les Sarrasins.}
\normalsize

\begin{verse}[\versewidth]
Olivier dit : Païens ont grande force,\\
Et nos Français, ce semble, en ont bien peu.\\
Ami Roland, sonnez de votre cor :\\
Charles l’entendra, et fera retourner son armée.\\
— Je serais bien fou, répond Roland ;\\
Dans la douce France, j’en perdrais ma gloire.\\
Non, mais je frapperai grands coups de Durendal ;\\
Le fer en sera sanglant jusqu’à l’or de la garde.\\
Félons païens furent mal inspirés de venir aux défilés :\\
Je vous jure que, tous, ils sont jugés à mort !

— Ami Roland, sonnez votre olifant :\\
Charles l’entendra et fera retourner la grande armée.\\
Le Roi et ses barons viendront à notre secours.\\
— À Dieu ne plaise, répond Roland,\\
Que mes parents jamais soient blâmés à cause de moi,\\
Ni que France la douce tombe jamais dans le déshonneur !\\
Non, mais je frapperai grands coups de Durendal,\\
Ma bonne épée, que j’ai ceinte à mon côté.\\
Vous en verrez tout le fer ensanglanté.\\
Félons païens sont assemblés ici pour leur malheur :\\
Je vous jure qu’ils seront tous livrés à mort !

— Ami Roland, sonnez votre olifant.\\
Le son en ira jusqu’à Charles qui passe aux défilés,\\
Et les Français, j’en suis certain, retourneront sur leurs pas.\\
— À Dieu ne plaise, lui répond Roland,\\
Qu’il soit jamais dit par aucun homme vivant\\
Que j’ai sonné mon cor à cause des païens !\\
Je ne ferai pas aux miens ce déshonneur.\\
Mais quand je serai dans la grande bataille,\\
J’y frapperai dix-sept cents coups :\\
De Durendal vous verrez le fer tout sanglant.\\
Français sont bons : ils frapperont en braves ;\\
Les Sarrasins ne peuvent échapper à la mort !

— Je ne vois pas où serait le déshonneur, dit Olivier.\\
J’ai vu, j’ai vu les Sarrasins d’Espagne ;\\
Les vallées, les montagnes en sont couvertes,\\
Les landes, toutes les plaines en sont cachées.\\
Qu’elle est puissante, l’armée de la gent étrangère,\\
Et que petite est notre compagnie !\\
— Tant mieux, répond Roland, mon ardeur s’en accroît :\\
Ne plaise à Dieu, ni à ses très-saints anges,\\
Que France, à cause de moi, perde de sa valeur !\\
Plutôt mourir qu’être déshonoré :\\
Plus nous frappons, plus l’Empereur nous aime !

Roland est preux, mais Olivier est sage ;\\
Ils sont tous deux de merveilleux courage.\\
Puis d’ailleurs qu’ils sont à cheval et en armes,\\
Ils aimeraient mieux mourir que d’esquiver la bataille.\\
Les comtes ont l’âme bonne, et leurs paroles sont élevées...\\
Félons païens chevauchent par grande ire :\\
Voyez un peu, Roland, dit Olivier ;\\
Les voici, les voici près de nous, et Charles est trop loin.\\
Ah ! vous n’avez pas voulu sonner de votre cor ;\\
Si le grand Roi était ici, nous n’aurions rien à craindre.\\
Jetez les yeux là-haut, vers les monts d’Espagne :\\
Vous y verrez dolente arrière-garde.\\
Tel s’y trouve aujourd’hui qui plus jamais ne sera dans une autre.\\
— Honteuse, honteuse parole, répond Roland.\\
Maudit soit qui porte un lâche cœur au ventre !\\
Nous tiendrons pied fortement sur la place :\\
De nous viendront les coups, et de nous la bataille !
\end{verse}

\begin{flushright}
\large
\textit{Chanson de Roland}, laisses \textsc{lxxxiii} à \textsc{lxxxvii}.

\medskip
\small
Traduction de Léon Gautier.
\end{flushright}
\end{document}
