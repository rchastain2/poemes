\documentclass[11pt,a4paper,french]{scrartcl}

\usepackage{babel}
\usepackage{microtype}

\usepackage{fontspec}
%\setmainfont[Ligatures=Discretionary]{EB Garamond}
%\setmainfont[Ligatures=Discretionary]{Palatino Linotype}
\setmainfont{EBGaramond}
[
  Extension      = .otf ,
  UprightFont    = *-Regular,
  ItalicFont     = *-Italic,
  BoldFont       = *-Bold,
  BoldItalicFont = *-BoldItalic,
  Numbers        = Lowercase,
  Ligatures      = Discretionary,
  Style          = Swash
]
%\defaultfontfeatures{Ligatures=TeX}
%\setmainfont{UnifrakturMaguntia}
%\setmainfont{Junicode}
%\setmainfont{TeX Gyre Schola}
%\setmainfont{TeX Gyre Bonum}
%\setmainfont{TeX Gyre Termes}
%\setmainfont{Libertinus Serif}
%\setmainfont{Erewhon}
%\setmainfont{Vollkorn}
%\setmainfont{Cabin}
%\setmainfont{Medieval Sharp}
%\setmainfont{TeX Gyre Pagella}
%\setmainfont{Latin Modern Roman}
%\defaultfontfeatures{Ligatures=TeX}
%\setmainfont[Numbers=OldStyle]{Junicode}

\usepackage{verse}
%\usepackage{showframe}

\begin{document}
\thispagestyle{empty}
\settowidth{\versewidth}{Frères humains, qui après nous vivez,}

% L'EPITAPHE
% EN FORME DE BALLADE
% Que feit Villon pour luy et ses compagnons, s'attendant
% estre pendu avec eulx.

%\poemtitle[]{%
%\begin{tabular}{c}
%L'EPITAPHE\\EN FORME DE BALLADE\\Que feit Villon pour luy et ses compagnons, s'attendant\\estre pendu avec eulx.
%\end{tabular}%
%}

\poemtitle[]{%
\begin{tabular}{c}L'EPITAPHE\end{tabular}\\
\medskip\small%
\begin{tabular}{c}EN FORME DE BALLADE\end{tabular}\\
\medskip\footnotesize%
\begin{tabular}{c}Que feit Villon pour luy et ses compagnons, s'attendant\\estre pendu avec eulx.\end{tabular}}

\begin{verse}[\versewidth]
\vin Frères humains, qui après nous vivez,\\
N'ayez les cueurs contre nous endurciz,\\
Car, si pitié de nous pouvres avez,\\
Dieu en aura plustost de vous merciz.\\
Vous nous voyez cy attachez cinq, six :\\
Quant de la chair, que nous avons nourrie,\\
Elle est pieça dévorée et pourrie,\\
Et nous, les os, devenons cendre et pouldre.\\
De nostre mal personne ne s'en rie,\\
Mais priez Dieu que tous nous vueille absouldre !

\vin Se vous clamons, frères, pas n'en devez\\
Avoir desdaing, quoyque fusmes occis\\
Par justice. Toutesfois, vous sçavez\\
Que tous les hommes n'ont pas bon sens assis ;\\
Intercédez doncques, de cueur rassis,\\
Envers le Filz de la Vierge Marie,\\
Que sa grace ne soit pour nous tarie,\\
Nous préservant de l'infernale fouldre.\\
Nous sommes morts, ame ne nous harie ;\\
Mais priez Dieu que tous nous vueille absouldre !

\vin La pluye nous a debuez et lavez,\\
Et le soleil dessechez et noirciz ;\\
Pies, corbeaux, nous ont les yeux cavés,\\
Et arrachez la barbe et les sourcilz.\\
Jamais, nul temps, nous ne sommes rassis ;\\
Puis çà, puis là, comme le vent varie,\\
A son plaisir sans cesser nous charie,\\
Plus becquetez d'oyseaulx que dez à coudre.\\
Ne soyez donc de nostre confrairie,\\
Mais priez Dieu que tous nous vueille absoudre !

\hspace{0.5\versewidth}{\poemtitlefont\small%
ENVOI.
}%

\vin Prince \textsc{Jesus}, qui sur tous seigneurie,\\
Garde qu'Enfer n'ayt de nous la maistrie :\\
A luy n'ayons que faire ne que souldre.\\
Hommes, icy n'usez de mocquerie\\
Mais priez Dieu que tous nous vueille absouldre !
\end{verse}

% https://gallica.bnf.fr/ark:/12148/bpt6k6468587z/f129.item.texteImage

\end{document}
