\documentclass[14pt,a4paper,french]{scrartcl}

\usepackage{verse}
\usepackage{babel}
\usepackage{microtype}

\usepackage{fontspec}
\setmainfont[Ligatures=Discretionary]{EB Garamond}
%\setmainfont[Ligatures=Discretionary]{Palatino Linotype}

% https://tex.stackexchange.com/a/123669/295527
%\frenchsetup{StandardItemLabels=true}

\begin{document}

\settowidth{\versewidth}{Un riche laboureur, sentant sa mort prochaine,}

\begin{verse}\poemtitle{Le Laboureur et ses Enfants.}
\vin Travaillez, prenez de la peine:\\
\vin C'est le fonds\footnotemark{} qui manque le moins.\\
Un riche laboureur, sentant sa mort prochaine,\\
Fit venir ses enfants, leur parla sans témoins.\\
Gardez-vous, leur dit-il, de vendre l'héritage\\
\vin Que nous ont laissé nos parents :\\
\vin Un trésor est caché dedans.\\
Je ne sais pas l'endroit ; mais un peu de courage\\
Vous le fera trouver : vous en viendrez à bout.\\
Remuez votre champ dès qu'on aura fait l'oût :\\
Creusez, fouillez, bêchez ; ne laissez nulle place\\
\vin Où la main ne passe et repasse.\\
Le père mort, les fils vous retournent le champ,\\
Deça, delà, partout ; si bien qu'au bout de l'an\\
\vin Il en rapporta davantage.\\
D'argent, point de caché. Mais le père fut sage\\
\vin De leur montrer, avant sa mort,\\
\vin Que le travail est un trésor.
\end{verse}

\begin{flushright}
\large
\textsc{La Fontaine}, \textit{Fables}.
\end{flushright}

\footnotetext{Un \textit{fonds}, c'est :
\begin{itemize}
\item\primo Le sol d'un champ, d'une terre, d'un domaine. Cultiver un fonds. Un bon fonds. Il ne faut pas bâtir sur le fonds d'autrui.
\item\secundo Par extension, une somme d'argent plus ou moins considérable destinée à quelque usage. Trouver un fonds. Dissiper un fonds. Avoir des fonds considérables.
\item\tertio Un bien, un capital quelconque, par opposition aux revenus qu'il produit.
\end{itemize}}

\end{document}
