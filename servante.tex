\documentclass[14pt,a4paper,french]{scrartcl}

\usepackage{verse}
\usepackage{babel}
\usepackage{microtype}

\usepackage{fontspec}
\setmainfont[Ligatures=Discretionary]{EB Garamond}
%\setmainfont[Ligatures=Discretionary]{Palatino Linotype}

\begin{document}
\thispagestyle{empty}

\settowidth{\versewidth}{La servante au grand cœur dont vous étiez jalouse,}

\begin{verse}\poemtitle{CXXIV}
La servante au grand cœur dont vous étiez jalouse,\\
Et qui dort son sommeil sous une humble pelouse,\\
Nous devrions pourtant lui porter quelques fleurs.\\
Les morts, les pauvres morts, ont de grandes douleurs,\\
Et quand Octobre souffle, émondeur des vieux arbres,\\
Son vent mélancolique à l’entour de leurs marbres,\\
Certe, ils doivent trouver les vivants bien ingrats,\\
De dormir, comme ils font, chaudement dans leurs draps,\\
Tandis que, dévorés de noires songeries,\\
Sans compagnon de lit, sans bonnes causeries,\\
Vieux squelettes gelés travaillés par le ver,\\
Ils sentent s’égoutter les neiges de l’hiver\\
Et le siècle couler, sans qu’amis ni famille\\
Remplacent les lambeaux qui pendent à leur grille.

Lorsque la bûche siffle et chante, si le soir,\\
Calme, dans le fauteuil je la voyais s’asseoir,\\
Si, par une nuit bleue et froide de décembre,\\
Je la trouvais tapie en un coin de ma chambre\\
Grave, et venant du fond de son lit éternel\\
Couver l’enfant grandi de son œil maternel,\\
Que pourrais-je répondre à cette âme pieuse,\\
Voyant tomber des pleurs de sa paupière creuse ?
\end{verse}

\begin{flushright}
\large
\textsc{Baudelaire}, \textit{Les Fleurs du mal}.
\end{flushright}

\end{document}
