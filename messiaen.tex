\documentclass[a5paper,12pt]{article}

\usepackage{fontspec}
\usepackage[french]{babel}
\usepackage{microtype}
\usepackage{geometry}
\geometry{margin=24mm}
\usepackage{comment}

\usepackage{dialogue}

\setmainfont{EBGaramond}
[
  Extension      = .otf ,
  UprightFont    = *-Regular,
  ItalicFont     = *-Italic,
  BoldFont       = *-Bold,
  BoldItalicFont = *-BoldItalic,
  Numbers        = Lowercase,
  Ligatures      = Discretionary,
  Style          = Swash
]
% https://texnique.fr/osqa/questions/8423/lualatex-et-ligatures

\begin{document}
\thispagestyle{empty}
\setlength\parindent{0pt}

\begin{comment}
PREMIER ACTE
Premier Tableau : La Croix
FRÈRE LEON
J'ai peur, sur la route, quand s'agrandissent et s'obscurcissent les fenêtres, quand ne rougissent plus les feuilles du poinsettia.
\textsc{saint françois}
Ô terre !... Ô ciel !... Frère Léon ?
FRÈRE LÉON
Mon père ?
\textsc{saint françois}
Même si le Frère Mineur rendait la vue aux aveugles, l'ouïe aux sourds, la parole aux muets : sache que tout cela n'est pas la joie, la joie parfaite.
\end{comment}

\begin{center}
\large
Troisième Tableau : Le Baiser au Lépreux
\end{center}

\begin{dialogue}
  \speak{le lépreux} Comment peut-on vivre une telle vie ? Tous ces Frères qui veulent me rendre service... S'ils enduraient ce que j'endure, s'ils souffraient ce que je souffre ! Ha !... Ha !... peut-être se révolteraient-ils à leur tour...
  \speak{saint françois} Dieu te donne la paix, frère bien-aimé !
  \speak{le lépreux} Quelle paix puis-je avoir de Dieu, qui m'a enlevé tout bien, m'a rendu tout pourri et fétide ?
  \speak{saint françois} Les infirmités du corps nous sont données pour le salut de notre âme. Comment comprendre la croix, si on n'en a pas porté un petit morceau ?
  \speak{le lépreux} J'en ai assez ! assez ! et plus qu'assez ! Les Frères que tu as mis à mon service, ils me soignent mal ! Au lieu de me soulager, ils m'infligent leurs horribles bavardages, leurs remèdes inutiles !
  \speak{saint françois} Et que fais-tu, ami, que fais-tu de la vertu, la vertu de patience ?
  \speak{le lépreux} Mais ce sont eux qui m'agacent, me bousculent dans tous les sens... et la démangeaison de mes pustules me rend fou...
  \speak{saint françois} Offre ton mal en pénitence, mon fils.
  \speak{le lépreux} La pénitence ! la pénitence ! Enlève-moi d'abord mes pustules, et après je ferai pénitence ! Et puis, tes Frères, je sais bien que je les dégoûte : quand ils me voient, ils ne retiennent même pas leur envie de vomir...
  \speak{saint françois} Pauvres Frères, ils font tout ce qu'ils peuvent...
  \speak{le lépreux} Autrefois, j'étais jeune et fort ! Maintenant, je suis comme une feuille frappée de mildiou : tout jaune, avec des taches noires...
  \speak{saint françois} Si l'homme intérieur est beau, il apparaîtra glorieux à l'heure de la résurrection.
  \speak{l'ange} Lépreux, ton cœur t'accuse.
  \speak{le lépreux} D'où vient cette voix ?
  \speak{saint françois} Écoute !...
  \speak{l'ange} Mais Dieu est plus grand que ton cœur.
  \speak{le lépreux} Qui est-ce qui chante ainsi ?
  \speak{saint françois} C'est peut-être un ange envoyé du ciel pour te réconforter...
  \speak{l'ange} Il est amour, il est plus grand que ton cœur, il connaît tout.
  \speak{le lépreux} Que dit-il ? Je ne comprends pas...
  \speak{saint françois} Il dit : \frquote{Ton cœur t'accuse, mais Dieu est plus grand que ton cœur.}
  \speak{l'ange} Mais Dieu est tout amour, et qui demeure dans l'amour demeure en Dieu, et Dieu en lui.
  \speak{le lépreux} Pardonne-moi, Père, je récrimine toujours... Tes Frères m'appellent le Lépreux !
  \speak{saint françois} Où se trouve la tristesse, que je chante la joie !
  \speak{le lépreux} Je sais bien que je suis horrible, et je me dégoûte moi-même...
  \speak{saint françois} Où se trouve l'erreur, que j'ouvre la vérité.
  \speak{le lépreux} Mais toi, tu es bon ! Tu m'appelles mon ami, mon frère, mon fils !
  \speak{saint françois} Où se trouvent les ténèbres, que j'apporte la lumière ! Pardonne-moi mon fils : je ne t'ai pas assez aimé.
  \speak{le lépreux} \direct{Saint Francois embrasse le Lépreux. Saint François s'écarte. Le Lépreux se tient debout, guéri, les bras levés, complètement transformé.} Miracle ! Regarde, Père, regarde : les taches ont disparu de ma peau ! Je suis guéri !
\end{dialogue}

\begin{flushright}
\large
\textsc{messiaen}, \textit{Saint François d'Assise.}
\end{flushright}
\end{document}
